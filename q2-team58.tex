\documentclass[a4paper]{exam}

\usepackage{amsmath}
\usepackage{amssymb}
\usepackage{amsthm}
\usepackage{array}
\usepackage{geometry}
\usepackage{hyperref}
\usepackage{titling}

\newcolumntype{C}{>{$}c<{$}} % math-mode version of "c" column type

\theoremstyle{definition}
\newtheorem{definition}{Definition}

\runningheader{CS/MATH 113}{WC05: Predicate Logic and Natural Language}{\theauthor}
\runningheadrule
\runningfootrule
\runningfooter{}{Page \thepage\ of \numpages}{}

\printanswers

\title{Weekly Challenge 05: Predicate Logic and Natural Language\\CS/MATH 113 Discrete Mathematics}
\author{q2-team-59}  % <== for grading, replace with your team name, e.g. q1-team-420
\date{Habib University | Spring 2023}

\qformat{{\large\bf \thequestion. \thequestiontitle}\hfill}
\boxedpoints

\begin{document}
\maketitle

\begin{questions}

  \titledquestion{Limit of a Function}
  \begin{definition}[Limit of a function at a point]
    Let $f(x)$ be a real valued function defined on an interval that contains $x=a$, except possibly at $x=a$. For $\epsilon$ and $\delta$ real numbers, we say that limit of the function $f(x)$ at $x=a$ is a real number $L$ and write $\lim_{x \to a} f(x)=L$ if 
    \[
      \forall \epsilon>0, \exists \delta>0 \ni (0<|x-a|<\delta \implies |f(x)-L|<\epsilon).
    \]
  \end{definition}
  The $\ni$ symbol denotes, ``such that''.
  \begin{parts}
  \item Express in English language, the definiton of the limit of a function at a point.
    \begin{solution}
      as the limit x approaches a the function becomes a real number L, which is for any positive $\epsilon$ there exists a positive $\delta$ such that (x-a) lies between 0 and $\delta$ and the difference between the function and L is less than $\epsilon$
    \end{solution}
  \item Express in English language, what it means for the limit of a function $f(x)$ to not exist at point $x=a$.
    \begin{solution}
      it means that the functions limit at x=a doesn't exist for every $\epsilon$ greater than 0, and there is no $\delta$ greater than 0 such that when modulus x-a is greater than 0 and greater than $\delta$, then modulos f(x)-L is less than $\epsilon$
    \end{solution}
  \item Express in predicate logic, what it means for the limit of a function $f(x)$ to not exist at point $x=a$.
    \begin{solution}
      
      $\exists$ $\epsilon$ $>$ 0 $\exists$ $\ni$ $\delta$ $>$ 0, ((0$<$$|$x-a$|$$<$$\delta$) $\land$ $\neg$ ( $|$f(x) -L$|$ ) $<$ $\epsilon$))
    \end{solution}
  \item If in the defintion of limit we swap the two quantifiers like this:
    \[
      \exists \delta>0 \ni \forall \epsilon>0, (|x-a|<\delta \implies |f(x)-L|<\epsilon).
    \]
    How does the new statement read? Does it still define the limit of a function? Discuss.
    \begin{solution}
      This statement does not define the limit of a function because it does not specify what the limit is - it only states the conditions that the limit must satisfy. The original definition states the conditions and also provides the value of the limit, so it is a more complete definition.
    \end{solution}
  \end{parts}
\end{questions}

\end{document}
